\documentclass[10pt]{beamer}
\usepackage{amsmath}
\usepackage{xfrac}
\usepackage{graphicx}
% Theme
\usetheme{Copenhagen}

% Title page
\title[Predator and Prey Model]{Predator and Prey Model}
\author[Mustafif Khan]{Mustafif Khan}
\institute[TMU]{MTH630-W2024}
\titlegraphic{
    \includegraphics[height=2cm]{tmu.jpg}
}
\date{\today}

\begin{document}

\begin{frame}
    \titlepage
\end{frame}

\begin{frame}{Outline}
    \tableofcontents
\end{frame}

\section{Lotka-Volterra Model}
\begin{frame}{Lotka-Volterra Model}
    The Lotka-Volterra Predator-Prey model consists of two first order
    non-linear differential equations. The model describes the interaction
    between
    two species, where one species is the predator $y$ and the other is the
    prey
    $x$.
    \begin{align}
        \frac{dx}{dt} & = ax - bxy = x(a-by)     \\
        \frac{dy}{dt} & = -cy + exy = y(-c + ex)
    \end{align}
    where $a$, $b$, $c$, and $e$ are $\geq 0$.
\end{frame}

\begin{frame}{Understanding the Parameters}
    Inside the equations, the parameters have the following meanings:latex

    \begin{itemize}
        \item $x$: the population of the prey species (e.g. rabbits)
        \item $y$: the population of the predator species (e.g. foxes)
        \item $a$: the natural growth rate of prey in the absence of predation
        \item $b$: the death rate per encounter of prey due to predation
        \item $c$: the natural death rate of the predator in the absence of
              prey
        \item $e$: the efficiency of turning predated prey into predator
              offspring
    \end{itemize}
\end{frame}

\section{Dimensionless Equations}
\begin{frame}{Dimensionless Equations}
    To simplify the equations, we can make them dimensionless.
    \begin{align*}
        u (\tau) = \cfrac{ex(t)}{c}  \\
        v (\tau) = -\cfrac{by(t)}{a} \\
        \tau = at
    \end{align*}
    Then we have the two following differential equations:
    \begin{align}
        \cfrac{du}{d\tau} & = u(1 - v)      \\
        \cfrac{dv}{d\tau} & = \alpha v(u-1)
    \end{align}
    Where $\alpha = \sfrac[diagonal]{c}{a}$.
\end{frame}

\section{Equilibrium Points}
\begin{frame}{Equilibrium Points}
    The equilibrium points of the system are found by setting the right hand
    side of the equations to zero. This gives us the following equilibrium
    points, $(0, 0)$ and $(1, 1)$.

    The Jacobian matrix of the system is given by:
    \begin{align*}
        J(u, v) = \begin{bmatrix}
                      1-v      & -u          \\
                      \alpha v & \alpha(u-1)
                  \end{bmatrix}
    \end{align*}

    The eigenvalues of the Jacobian matrix at the equilibrium points are
    $J(0, 0) = \begin{bmatrix}
            1 & 0       \\
            0 & -\alpha
        \end{bmatrix}$ and $J(1, 1) = \begin{bmatrix}
            0      & -1 \\
            \alpha & 0
        \end{bmatrix}$

    The eigenvalues of $J(0, 0)$ are $1$ and $-\alpha$, and the eigenvalues of
    $J(1, 1)$ are $\pm i\sqrt{\alpha}$.
\end{frame}

\section{Stability Analysis}
\begin{frame}{Stability Analysis}
    Since the eigenvalues of $J(0, 0)$ are both real and that $\lambda_1
        \lambda_2 < 0$, the equilibrium point $(0, 0)$ is a saddle point.
    \\
    Since both of the eigenvalues of $J(1, 1)$ are complex, the equilibrium
    point $(1, 1)$ is a center, or stable/unstable spiral.

    \begin{itemize}
        \item The equilibrium point $(0, 0)$ represents when both species are
              extinct.
        \item The equilibrium point $(1, 1)$ represents when both species
              coexist in time.
        \item A center can lead to a regular, cyclic behavior in both
              populations that oscillate around the equilibrium point in a stable manner.
        \item A stable spiral indicates that there can be a perturbation in one
              of the populations, but they will eventually return to the equilibrium state.
        \item An unstable spiral suggests that the populations can experience
              population collapses or rapid growth due to a small perturbation.
    \end{itemize}
\end{frame}

\begin{frame}
    Consider the following example, where we have the following values for the
    constants:
    $a = 0.1$, $b = 0.02$, $c = 0.2$, and $e = 0.01$. We then get the following
    dimensionless equations:

    \begin{align*}
        \cfrac{du}{d\tau} & = u(1 - v) \\
        \cfrac{dv}{d\tau} & = 2v(u-1)
    \end{align*}
    This has the following Jacobian matrices at the equilibrium points:

    \begin{align*}
        J(0, 0) = \begin{bmatrix}
                      1 & 0  \\
                      0 & -2
                  \end{bmatrix}
        \quad
        J(1, 1) = \begin{bmatrix}
                      0 & -1 \\
                      2 & 0
                  \end{bmatrix}
    \end{align*}
\end{frame}

\begin{frame}
    The Jacobian matrix at $(0, 0)$ has eigenvalues $1$ and $-2$, and the has the
    following phase plane analysis:
    \begin{itemize}
        \item Trace = $1 - 2 = -1$
        \item Determinant = $1 \cdot -2 = -2$
        \item Discriminant = $(-1)^2 - 4(-2) = 1 + 8 = 9$
    \end{itemize}

    Since the discriminant is positive and the determinant is negative, the
    equilibrium point $(0, 0)$ is a saddle point.
\end{frame}

\begin{frame}
    The Jacobian matrix at $(1, 1)$ has eigenvalues $\pm i\sqrt{2}$, and the phase
    plane analysis is as follows:
    \begin{itemize}
        \item Trace = $0$
        \item Determinant = $0 - (-2) = 2$
        \item Discriminant = $0^2 - 8 = -8$
    \end{itemize}

    Since the discriminant is negative and the trace is zero, the equilibrium point
    $(1, 1)$ is a center.

\end{frame}

\end{document}
